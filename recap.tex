\documentclass{article}

\usepackage[letterpaper, portrait, margin=0.8in]{geometry}

\usepackage{amsmath, amsfonts, amsthm, amssymb}
\usepackage{graphicx, float}
\usepackage{mathtools}
\usepackage{siunitx}
\usepackage{esdiff}
\usepackage{titlesec}
\usepackage{hyperref}
\usepackage{interval}
\usepackage{mdframed}
\usepackage{multicol}

%opening
\title{PCH Winter Summary}
\author{Jayden Li}
\date{December 2023 -- February 2024}

\raggedcolumns

\begin{document}
\maketitle

\linespread{1.25}
\fontsize{12pt}{12pt}\selectfont

\section{Trigonometry}
\subsection{Pythagorean Identity}
\begin{center}
	$\sin^2x+\cos^2x=1$
\end{center}

\subsection{Even/Odd Identities}
\begin{center}
	Cosine is even. $\cos\left(-\theta\right)=\cos\theta$ \\
	Sine is odd. $\sin\left(-\theta\right)=-\sin\theta$ \\
	Tangent is odd. $\tan\left(-\theta\right)=-\tan\theta$ \\
\end{center}

\subsection{Angle Addition and Subtraction Identities}
\begin{gather*}
	\sin\left(\alpha+\beta\right)=
	\sin\alpha\cos\beta+\cos\alpha\sin\beta \\
	\sin\left(\alpha-\beta\right)=
	\sin\alpha\cos\beta-\cos\alpha\sin\beta \\
	\cos\left(\alpha+\beta\right)=
	\cos\alpha\cos\beta-\sin\alpha\sin\beta \\
	\cos\left(\alpha-\beta\right)=
	\cos\alpha\cos\beta+\sin\alpha\sin\beta \\
	\hyperref[proof:tanadd]{
		\tan\left(\alpha+\beta\right)=
		\frac{\tan\alpha+\tan\beta}{1-\tan\alpha\tan\beta}
		} \\
	\hyperref[proof:tansub]{
		\tan\left(\alpha-\beta\right)=
		\frac{\tan\alpha-\tan\beta}{1+\tan\alpha\tan\beta}
	} \\
\end{gather*}

\subsection{Double Angle Identites (PS \#29)}
\begin{gather*}
	\hyperref[proof:sindouble]{
		\sin2\theta=2\sin\theta\cos\theta
	} \\
	\hyperref[proof:cosdouble]{
		\begin{aligned}
			\cos2\theta&=\cos^2\theta-\sin^2\theta \\
			&=1-2\sin^2\theta \\
			&=2\cos^2\theta-1 \\
		\end{aligned}
	} \\
	\hyperref[proof:tandouble]{
		\tan2\theta=\frac{2\tan\theta}{1-\tan^2\theta}
	} \\
\end{gather*}

\subsection{Half Angle Identites (PS \#30)}
\begin{gather*}
	\hyperref[proof:sinhalf]{
		\sin\frac{\theta}{2}=\pm\sqrt{\frac{1-\cos\theta}{2}}
	} \\
	\hyperref[proof:coshalf]{
		\cos\frac{\theta}{2}=\pm\sqrt{\frac{1+\cos\theta}{2}}
	} \\
	\hyperref[proof:tandouble]{
		\begin{aligned}
		\tan\frac{\theta}{2}
		&=\pm\sqrt{\frac{1-\cos\theta}{1+\cos\theta}} \\
		&=\frac{1-\cos\theta}{\sin\theta} \\
		&=\frac{\sin\theta}{1+\cos\theta} \\
		\end{aligned}
	} \\
\end{gather*}

\subsection{Product-to-Sum Identities (PS \#31)}
\begin{center}
\hyperref[proof:p2s]{
	$\begin{aligned}
		\sin\alpha\cos\beta=\frac{\sin\left(\alpha+\beta\right)
			+\sin\left(\alpha-\beta\right)}{2} \\
		\cos\alpha\sin\beta=\frac{\sin\left(\alpha+\beta\right)
			-\sin\left(\alpha-\beta\right)}{2} \\
			\cos\alpha\cos\beta=\frac{\cos\left(\alpha-\beta\right)
				+\cos\left(\alpha+\beta\right)}{2} \\
		\sin\alpha\sin\beta=\frac{\cos\left(\alpha-\beta\right)
			-\cos\left(\alpha+\beta\right)}{2} \\
	\end{aligned}$
}
\end{center}

\subsection{Sum-to-Product Identities (PS \#31)}
\begin{center}
\hyperref[proof:s2p]{
	$\begin{aligned}
		\sin\alpha+\sin\beta&=
		2\sin\frac{\alpha+\beta}{2}\cos\frac{\alpha-\beta}{2} \\
		\sin\alpha-\sin\beta&=
		2\cos\frac{\alpha+\beta}{2}\sin\frac{\alpha-\beta}{2} \\
		\cos\alpha+\cos\beta&=
		2\cos\frac{\alpha+\beta}{2}\cos\frac{\alpha-\beta}{2} \\
		\cos\alpha-\cos\beta&=
		-2\sin\frac{\alpha+\beta}{2}\sin\frac{\alpha-\beta}{2} \\
	\end{aligned}$
}
\end{center}

\pagebreak

\subsection{Finding Range}
\subsubsection{Quadratic with Trig Functions}
Complete the square and isolate the trig function. Example (Koch CYU \#1, Q4):
\begin{multicols}{2}
	\begin{align*}
		y&=3\cos x+2\sin^2x+1 \\
		&=3\cos x+2\left(1-\cos^2x\right)+1 \\
		&=-2\cos^2x+3\cos x+3 \\
		&=-2\left(\cos^2x-\frac{3}{2}\cos x-\frac{3}{2}\right) \\
		&=-2\left(\left(\cos x-\frac{3}{4}\right)^2-\frac{9}{16}-\frac{3}{2}\right) \\
		&=-2\left(\cos x-\frac{3}{4}\right)^2+\frac{33}{8}
	\end{align*}
	\begin{gather*}
		-1\leq\cos x\leq1 \\
		-\frac{7}{4}\leq\cos x-\frac{3}{4}\leq\frac{1}{4} \\
		0\leq\left(\cos x-\frac{3}{4}\right)^2\leq\frac{49}{16} \\
		-\frac{49}{8}\leq-2\left(\cos x-\frac{3}{4}\right)^2\leq0 \\
		-\frac{16}{8}\leq-2\left(\cos x-\frac{3}{4}\right)^2+\frac{33}{8}\leq\frac{33}{8} \\
		-2\leq y\leq\frac{33}{8} \\
		\boxed{\text{Range: $\interval[scaled]{-2}{\frac{33}{8}}$}}
	\end{gather*}
\end{multicols}
\subsubsection{Addition of Sine and Cosine}
Rewrite $y=a\sin x+b\cos x$ in the form $y=r\cos\left(x-a\right)$. Example (PS \#31, Q8a):
\begin{multicols}{2}
\begin{align*}
	y&=\sqrt{3}\sin x-\cos x \\
	y&=r\cos\left(x-a\right) \\
	y&=r\cos x\cos a+r\sin x\sin a
\end{align*}
\begin{align*}
	\intertext{By matching coefficients:}
	r\sin a&=-1 \\
	r\cos a&=\sqrt{3} \\
	\intertext{Square both equations:}
	r^2\sin^2a&=1 \\
	r^2\cos^2a&=3 \\
	\intertext{Apply Pythagorean Identity:}
	r^2\sin^2a&=1 \\
	r^2-r^2\sin^2a&=3 \\
	\intertext{Add the two equations:}
	r^2&=4 \\
	r=\pm2
\end{align*}
Because the range of the cosine function is always
$\interval[scaled]{-1}{1}$, the range of $r\cos\left(x-a\right)$
is $\interval[scaled]{-r}{r}$. No matter what the sign of $r$ is,
the range is \boxed{\interval[scaled]{-2}{2}}.

\end{multicols}

\subsection{Derivatives of Trigonometric Functions (PS \#32)}
\subsection{Trigonometric Limits (PS \#33)}
\subsection{Inverse Trigonometric Functions (PS \#34-35)}





\pagebreak
\section{Derivations}

\subsection{Tangent Angle Addition}
\label{proof:tanadd}
\begin{proof}
	$\begin{aligned}[t]
		\tan\left(\alpha+\beta\right)
		&=\frac{\sin\left(\alpha+\beta\right)}
			 {\cos\left(\alpha+\beta\right)} \\
		&=\frac{\sin\alpha\cos\beta+\cos\alpha\sin\beta}
			 {\cos\alpha\cos\beta-\sin\alpha\sin\beta} \\
		&=\frac{\sin\alpha\cos\beta+\cos\alpha\sin\beta}{\cos\alpha\cos\beta}/
		\frac{\cos\alpha\cos\beta-\sin\alpha\sin\beta}{\cos\alpha\cos\beta} \\
		&=\left(\frac{\sin\alpha\cos\beta}{\cos\alpha\cos\beta}+
		\frac{\cos\alpha\sin\beta}{\cos\alpha\cos\beta}\right)/
		\left(\frac{\cos\alpha\cos\beta}{\cos\alpha\cos\beta}-
		\frac{\sin\alpha\sin\beta}{\cos\alpha\cos\beta}\right) \\
		&=\boxed{\frac{\tan\alpha+\tan\beta}{1-\tan\alpha\tan\beta}} \\
	\end{aligned} \\$
\end{proof}

\subsection{Tangent Angle Subtraction}
\label{proof:tansub}
\begin{proof}
	$\begin{aligned}[t]
		\tan\left(\alpha-\beta\right)
		&=\tan\left(\alpha+\left(-\beta\right)\right) \\
		&=\frac{\tan\alpha+\tan\left(-\beta\right)}
			 {1-\tan\alpha\tan\left(-\beta\right)} \\
		&=\frac{\tan\alpha+\left(-\tan\beta\right)}
			 {1-\tan\alpha\cdot\left(-\tan\beta\right)} \\
		&=\boxed{\frac{\tan\alpha-\tan\beta}{1+\tan\alpha\tan\beta}} \\
	\end{aligned} \\$
\end{proof}

\subsection{Sine Double Angle}
\label{proof:sindouble}
\begin{proof}
	$\begin{aligned}[t]
		\sin2\theta&=\sin\left(\theta+\theta\right) \\
		&=\sin\theta\cos\theta+\cos\theta\sin\theta \\
		&=\boxed{2\sin\theta\cos\theta} \\
	\end{aligned} \\$
\end{proof}

\subsection{Cosine Double Angle}
\label{proof:cosdouble}
\begin{proof}
	$\begin{aligned}[t]
		\cos2\theta&=\cos\left(\theta+\theta\right) \\
		&=\cos\theta\cos\theta-\sin\theta\sin\theta \\
		&=\boxed{\cos^2\theta-\sin^2\theta} \\
		&=\cos^2\theta-\left(1-\cos^2\theta\right) \\
		&=\boxed{2\cos^2\theta-1} \\
		&=2\left(1-\sin^2\theta\right)-1 \\
		&=\boxed{1-2\sin^2\theta} \\
	\end{aligned} \\$
\end{proof}

\subsection{Tangent Double Angle}
\label{proof:tandouble}
\begin{proof}
	$\begin{aligned}[t]
		\tan2\theta&=\tan\left(\theta+\theta\right) \\
		&=\frac{\tan\theta+\tan\theta}{1-\tan\theta\tan\theta} \\
		&=\boxed{\frac{2\tan\theta}{1-\tan^2\theta}} \\
	\end{aligned} \\$
\end{proof}

\subsection{Sine Half Angle}
\label{proof:sinhalf}
\begin{proof}
	$\begin{aligned}[t]
		\cos\theta&=\cos\left(2\cdot\frac{\theta}{2}\right) \\
		\cos\theta&=1-2\sin^2\frac{\theta}{2} \\
		2\sin^2\frac{\theta}{2}&=1-\cos\theta \\
		\Aboxed{\sin\frac{\theta}{2}&=\pm\sqrt{\frac{1-\cos\theta}{2}}} \\
	\end{aligned} \\$
\end{proof}

\subsection{Cosine Half Angle}
\label{proof:coshalf}
\begin{proof}
	$\begin{aligned}[t]
		\cos\theta&=\cos\left(2\cdot\frac{\theta}{2}\right) \\
		\cos\theta&=2\cos^2\frac{\theta}{2}-1 \\
		2\cos^2\frac{\theta}{2}&=1+\cos\theta \\
		\Aboxed{\cos\frac{\theta}{2}&=\pm\sqrt{\frac{1+\cos\theta}{2}}} \\
	\end{aligned} \\$
\end{proof}

\subsection{Tangent Half Angle}
\label{proof:tanhalf}
\begin{proof}
	$\begin{aligned}[t]
		\tan\frac{\theta}{2}&=\sin\frac{\theta}{2}/{\cos\frac{\theta}{2}} \\
		&=\sqrt{\frac{1-\cos\theta}{2}}/\sqrt{\frac{1+\cos\theta}{2}} \\
		&=\sqrt{\frac{1-\cos\theta}{2}/\frac{1+\cos\theta}{2}} \\
		&=\boxed{\pm\sqrt{\frac{1-\cos\theta}{1+\cos\theta}}} \\
		&=\sqrt{\frac{1-\cos\theta}{1+\cos\theta}\cdot\frac{1-\cos\theta}{1-\cos\theta}} \\
		&=\sqrt{\frac{\left(1-\cos\theta\right)^2}{1-\cos^2\theta}} \\
		&=\sqrt{\frac{\left(1-\cos\theta\right)^2}{\sin^2\theta}} \\
		&=\boxed{\frac{1-\cos\theta}{\sin\theta}} \\
		&=\sqrt{\frac{1-\cos\theta}{1+\cos\theta}\cdot\frac{1+\cos\theta}{1+\cos\theta}} \\
		&=\sqrt{\frac{1-\cos^2\theta}{\left(1-\cos\theta\right)^2}} \\
		&=\sqrt{\frac{\sin^2\theta}{\left(1+\cos\theta\right)^2}} \\
		&=\boxed{\frac{\sin\theta}{1+\cos\theta}} \\
	\end{aligned} \\$
\end{proof}

\subsection{Product-to-Sum Identities}
\label{proof:p2s}
\begin{proof}
	\begin{align}
		\sin\left(\alpha+\beta\right)=
		\sin\alpha\cos\beta+\cos\alpha\sin\beta \\
		\sin\left(\alpha-\beta\right)=
		\sin\alpha\cos\beta-\cos\alpha\sin\beta \\
		\cos\left(\alpha+\beta\right)=
		\cos\alpha\cos\beta-\sin\alpha\sin\beta \\
		\cos\left(\alpha-\beta\right)=
		\cos\alpha\cos\beta+\sin\alpha\sin\beta
	\end{align}
	\begin{align*}
		\intertext{\centering{$(1)+(2)$}}
		\sin\left(\alpha+\beta\right)+\sin\left(\alpha-\beta\right)&=
		\sin\alpha\cos\beta+\cos\alpha\sin\beta
		+\sin\alpha\cos\beta-\cos\alpha\sin\beta \\
		\sin\left(\alpha+\beta\right)+\sin\left(\alpha-\beta\right)&=
		2\sin\alpha\cos\beta \\
		\Aboxed{\sin\alpha\cos\beta&=\frac{\sin\left(\alpha+\beta\right)
		+\sin\left(\alpha-\beta\right)}{2}}
	\end{align*}
	\begin{align*}
		\intertext{\centering{$(1)-(2)$}}
		\sin\left(\alpha+\beta\right)-\sin\left(\alpha-\beta\right)&=
		\sin\alpha\cos\beta+\cos\alpha\sin\beta
		-\sin\alpha\cos\beta+\cos\alpha\sin\beta \\
		\sin\left(\alpha+\beta\right)-\sin\left(\alpha-\beta\right)&=
		2\cos\alpha\sin\beta \\
		\Aboxed{\cos\alpha\sin\beta&=\frac{\sin\left(\alpha+\beta\right)
		-\sin\left(\alpha-\beta\right)}{2}}
	\end{align*}
	\begin{align*}
		\intertext{\centering{$(3)+(4)$}}
		\cos\left(\alpha+\beta\right)+\cos\left(\alpha-\beta\right)&=
		\cos\alpha\cos\beta-\sin\alpha\sin\beta
		+\cos\alpha\cos\beta+\sin\alpha\sin\beta \\
		\cos\left(\alpha+\beta\right)+\cos\left(\alpha-\beta\right)&=
		2\cos\alpha\cos\beta \\
		\Aboxed{\cos\alpha\cos\beta&=\frac{\cos\left(\alpha-\beta\right)
		+\cos\left(\alpha+\beta\right)}{2}}
	\end{align*}
	\begin{align*}
		\intertext{\centering{$(3)-(4)$}}
		\cos\left(\alpha+\beta\right)-\cos\left(\alpha-\beta\right)&=
		\cos\alpha\cos\beta-\sin\alpha\sin\beta
		-\cos\alpha\cos\beta-\sin\alpha\sin\beta \\
		\cos\left(\alpha+\beta\right)-\cos\left(\alpha-\beta\right)&=
		-2\sin\alpha\sin\beta \\
		\sin\alpha\sin\beta&=-\frac{\cos\left(\alpha+\beta\right)
		-\cos\left(\alpha-\beta\right)}{2} \\
		\Aboxed{\sin\alpha\sin\beta&=\frac{\cos\left(\alpha-\beta\right)
		-\cos\left(\alpha+\beta\right)}{2}}
	\end{align*}
\end{proof}

\subsection{Sum-to-Product Identities}
\label{proof:s2p}
\begin{proof}
	\begin{align*}
		\sin\alpha\cos\beta=\frac{\sin\left(\alpha+\beta\right)
			+\sin\left(\alpha-\beta\right)}{2} \\
		\cos\alpha\sin\beta=\frac{\sin\left(\alpha+\beta\right)
			-\sin\left(\alpha-\beta\right)}{2} \\
		\cos\alpha\cos\beta=\frac{\cos\left(\alpha-\beta\right)
			+\cos\left(\alpha+\beta\right)}{2} \\
		\sin\alpha\sin\beta=\frac{\cos\left(\alpha-\beta\right)
			-\cos\left(\alpha+\beta\right)}{2}
	\end{align*}
	\begin{gather*}
		\text{Let }\theta=\alpha+\beta,\,\varphi=\alpha-\beta \\
		\alpha=\frac{\theta+\varphi}{2},\,\beta=\frac{\theta-\varphi}{2}
	\end{gather*}
	\begin{align*}
		\sin\frac{\theta+\varphi}{2}\cos\frac{\theta-\varphi}{2}
		&=\frac{\sin\theta+\sin\varphi}{2} \\
		\cos\frac{\theta+\varphi}{2}\sin\frac{\theta-\varphi}{2}
		&=\frac{\sin\theta-\sin\varphi}{2} \\
		\cos\frac{\theta+\varphi}{2}\cos\frac{\theta-\varphi}{2}
		&=\frac{\cos\varphi+\cos\theta}{2} \\
		\sin\frac{\theta+\varphi}{2}\sin\frac{\theta-\varphi}{2}
		&=\frac{\cos\varphi-\cos\theta}{2}
	\end{align*}
	\begin{align*}
		\Aboxed{\sin\theta+\sin\varphi&=
		2\sin\frac{\theta+\varphi}{2}\cos\frac{\theta-\varphi}{2}} \\
		\Aboxed{\sin\theta-\sin\varphi&=
		2\cos\frac{\theta+\varphi}{2}\sin\frac{\theta-\varphi}{2}} \\
		\cos\varphi+\cos\theta&=
		2\cos\frac{\theta+\varphi}{2}\cos\frac{\theta-\varphi}{2} \\
		\cos\varphi-\cos\theta&=
		2\sin\frac{\theta+\varphi}{2}\sin\frac{\theta-\varphi}{2} \\
	\end{align*}
	\begin{align*}
		\Aboxed{\cos\theta+\cos\varphi&=
		2\cos\frac{\theta+\varphi}{2}\cos\frac{\theta-\varphi}{2}} \\
		\Aboxed{\cos\theta-\cos\varphi&=
		-2\sin\frac{\theta+\varphi}{2}\sin\frac{\theta-\varphi}{2}} \\
	\end{align*}
\end{proof}


\end{document}
